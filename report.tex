\documentclass[pdftex,ptm,12pt,a4paper]{report}
\renewcommand{\baselinestretch}{1.5}
\setcounter{secnumdepth}{5}
\usepackage[english, russian]{babel}

% PDF search & cut'n'paste
\usepackage{cmap}
\usepackage[table,xcdraw]{xcolor}
\renewcommand{\baselinestretch}{1.5}
\usepackage{setspace}
\usepackage{indentfirst}

% Cyrillic support
\usepackage{mathtext}
\usepackage{amsmath}
\usepackage[T1,T2A]{fontenc}
\DeclareSymbolFont{T2Aletters}{T2A}{cmr}{m}{it}
\usepackage[utf8]{inputenc}
\usepackage{multicol}

\usepackage[bottom=30mm,top=20mm,right=20mm,left=30mm,headsep=0cm,nofoot]{geometry}

\usepackage{calc}
\setlength{\footskip}{\paperheight
  -(1in+\voffset+\topmargin+\headheight+\headsep+\textheight)
  -0.75in}

\usepackage{array}
\newcolumntype{C}[1]{>{\centering\let\newline\\\arraybackslash\hspace{0pt}}m{#1}}

\makeatletter
\renewcommand*{\ps@plain}{%
  \let\@mkboth\@gobbletwo
  \let\@oddhead\@empty
  \def\@oddfoot{%
    \reset@font
    \hfil
    \thepage
    % \hfil % removed for aligning to the right
  }%
  \let\@evenhead\@empty
  \let\@evenfoot\@oddfoot
}
\makeatother
\pagestyle{plain}

\usepackage[pdftex]{graphicx}
\usepackage{caption}
\usepackage{subcaption}
\usepackage[russian,english]{babel}
    \addto{\captionsenglish}{\renewcommand{\bibname}{Литература}}
    \addto\captionsenglish{\renewcommand{\figurename}{Рис.}}
    \addto\captionsenglish{\renewcommand{\contentsname}{Содержание}}
    \addto\captionsenglish{\renewcommand{\proofname}{Доказательство}}
\usepackage{hyperref}
\usepackage{cleveref}
\usepackage{url}
\usepackage{abstract}
\usepackage{float}
\usepackage{amsthm}
\usepackage{amssymb}
\usepackage{amsmath}
\renewcommand*{\proofname}{Доказательство}
\usepackage{indentfirst}
\usepackage{color}
\usepackage{natbib}
\usepackage{bbm, dsfont}


% Detect whether PDFLaTeX is in use
\usepackage{ifpdf}

% Fix links to floats
\usepackage[all]{hypcap}

\makeatletter
\renewcommand{\@chapapp}{Часть}
\makeatother

% Theorem Styles
\newtheorem{theorem}{Теорема}[chapter]
\newtheorem{lemma}[theorem]{Лемма}
\newtheorem{claim}[theorem]{Теорема}
% Definition Styles
\theoremstyle{definition}
\newtheorem{definition}{Определение}[chapter]
\newtheorem{example}{Пример}[chapter]
% Rule for Title Page
\newcommand{\HRule}{\rule{\linewidth}{0.5mm}}

\begin{document}

\begin{titlepage}
\newpage

\begin{center}
МИНИСТЕРСТВО ОБРАЗОВАНИЯ И НАУКИ РОССИЙСКОЙ ФЕДЕРАЦИИ \\
\vspace{0.5cm}
ГОСУДАРСТВЕННОЕ ОБРАЗОВАТЕЛЬНОЕ УЧРЕЖДЕНИЕ \\*
ВЫСШЕГО ПРОФЕССИОНАЛЬНОГО ОБРАЗОВАНИЯ\\*
"МОСКОВСКИЙ ФИЗИКО-ТЕХНИЧЕСКИЙ ИНСТИТУТ \\*
(ГОСУДАРСТВЕННЫЙ УНИВЕРСИТЕТ)" \\*
\vspace{0.5cm}
ФАКУЛЬТЕТ ИННОВАЦИЙ И ВЫСОКИХ ТЕХНОЛОГИЙ \\*
КАФЕДРА АНАЛИЗА ДАННЫХ \\*
\hrulefill
\end{center}


\vspace{3em}

\begin{center}
\Large Выпускная квалификационная работа по направлению 01.03.02 <<Прикладная математика и информатика>> \linebreak НА ТЕМУ:
\end{center}

\vspace{2.5em}

\begin{center}
\textsc{\large{\textbf{Алгоритмы консенсуса на энергонезависимой памяти произвольного доступа}}}
\end{center}

\vspace{6.5em}

\begin{minipage}{.45\linewidth}
\begin{flushleft}
Студент \\ Научный руководитель к.ф-м.н \\ Зам. зав. кафедрой д.ф-м.н, проф.
\end{flushleft}
\end{minipage}
\hfill
\begin{minipage}{.45\linewidth}
\begin{flushright}
    Сурин М.С.\\ Бабенко А.В.\\Бунина Е.И.
\end{flushright}
\end{minipage}

\vspace{\fill}

\begin{center}
МОСКВА, 2017
\end{center}

\end{titlepage}

\tableofcontents
\sloppy

\setstretch{1.5}
\parindent=1.25cm

\chapter{Введение}
\section{NVDIMM}
Дизайн СУБД всегда был вынужден принимать во внимание принципиальную разницу между энергозависимыми устройствами хранения информации и энергонезависимыми.
Разница проявляется как в латентности доступа, так и в особенностях предоставляемого интерфейса. Энергонезависимые устройства как правило предоставляют блочный интерфейс, то есть
позволяют оперировать только блоками из тысяч байт. В то время как современные DIMM дают возможность оперировать словами -- единицами байт.
Но не так давно (2014-15гг) появились на рынке технологии под общим названием “NVDIMM”, занимающие некое промежуточное звено сохраняя данные при потере питания, предоставляя скорость доступа, сравнимую с DRAM и также давая возможность адресации отдельных байтов.

Большая часть продуктов на 2014г (например продукты Miron Technology) представляют из себя модуль DRAM служащий кешeм для модуля энергонезависимой памяти (как правило NAND FLASH)
и автономным источником питания. Но тогда же были анонсированы, а позже и появились на рынке устройства без подобного разделения -- к примеру Intel Optane NVDIMM \cite{peng2019system}.

Порядки времён доступа к промышленным устройствам:
\begin{center}
\begin{tabular} {|l| c c c|}
\hline
    & SSD & NVDIMM & DIMM \\
    \hline
чтение & 200us & 120ns & 80ns \\
запись & 2ms & 750ns & 80ns \\
\hline
\end{tabular}
\end{center}

Такая небольшая разница между быстрой оперативной памятью и медленной энергонезависимой даёт возможность пересмотреть традиционные подходы к дизайну и архитектуре СУБД в перспективе давая возможность добиться большей производительности.

\section{Обзор литературы}
Несмотря на отсутствие на рынке устройств, исследователи давно пытаются оптимизировать алгоритмы \cite{iwabuchi2014nvm} и структуры данных \cite{chen2015persistent} для персистентной памяти .
Многие из них пользуются либо программными платформами для эмуляции задержек NVDIMM \cite{sengupta2015framework} либо занимаются построением аппаратных моделей \cite{dong2012nvsim}.
Использование персистентной памяти ставит перед исследователями новые задачи, такие как менеджмент памяти \cite{schwalb2015nvm}. Также своя специфика есть у задачи обеспечения
целостности данных при потере питания, так как большинство современных процессоров кешируют доступы к памяти и практически не предоставляют примитивов для управления своим кешем.
Как следствие для работы с персистентной памятью появились отдельные библиотеки такие как PMDK \cite{pmdk} и api \cite{kolli2016delegated}.
Задачей построения архитектуры СУБД, занимается удивительно малое количество исследователей и в основном можно выделить исследователей из Carnegie Mellon School of Computer Science
\cite{pavlo17}, \cite{arulraj2015let}, \cite{debrabant2014prolegomenon}, \cite{arulraj2017build}, \cite{writebehind}.  В \cite{writebehind} приведeна альтернатива write-ahead logging
для NVM, позволяющая повысить производительность СУБД и рассмотрена задача репликации с учётом NVM, но задача координации не рассматривалась.
Задачей исследования алгоритмов консенсуса для NVM по нашим сведениям никто не занимался.

\chapter{Цель работы}


\chapter{Заключение}

\bibliography{report}
\bibliographystyle{plain}


\end{document}
